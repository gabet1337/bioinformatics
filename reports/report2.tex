\documentclass[a4paper,oneside,article,11pt]{memoir}
\usepackage{changepage}
\usepackage[english]{babel}
\usepackage[utf8]{inputenc}
\usepackage{amsmath,amssymb,amsthm}

\newtheorem*{toprove}{At bevise}
\newtheorem{mydef}{Definition}

% This font looks so good.
\usepackage[sc]{mathpazo}
\usepackage{fancybox}
% Typesetting pseudo-code
\usepackage{algorithm}
\usepackage{algorithmic}
\usepackage{multirow}
% Code comments like [CLRS]
\renewcommand{\algorithmiccomment}[1]{\makebox[5cm][l]{$\triangleright$ \textit{#1}}}
\usepackage{framed,graphicx,xcolor}
\usepackage{listings}
\usepackage{hyperref}

\usepackage[font={small,it}]{caption}

% Relative references
\usepackage{varioref}

\usepackage{tikz}

\bibliographystyle{plain}

\title{Bioinformatics - Tree Comparison \\ Project 2}
\author{Peter Gabrielsen 20114179\\
Christoffer Hansen 20114637}
\newcounter{qcounter}

\makeatletter
\newenvironment{CenteredBox}{% 
\begin{Sbox}}{% Save the content in a box
\end{Sbox}\centerline{\parbox{\wd\@Sbox}{\TheSbox}}}% And output it centered
\makeatother

\begin{document}

\maketitle

\chapter*{Introduction}
%A short status of your work. Does everything work as expected, or are there any problems or unsolved issues.
In this project we are going to implement Saitou and Nei's algorithm for Neighbor joining. The aim of the project will be to make it run as fast as possible using different techniques and then compare our fastest implementation against \texttt{RapidNJ} and \texttt{QuickTree}.
Our implementation works as expected.
\\\\Code can be found at \url{URL HERE}. %TODO

\pagebreak

\chapter*{Implementation}
The implementations is done in \texttt{c++}. It compiles using \texttt{make nj} and runs using \texttt{./nj $<$phylib file$>$}.
We used OpenMP to parallelize our algorithm.

\chapter*{Machine}
%TODO insert the machinery here!

\chapter*{Results}
\begin{figure}[H]
\begin{adjustwidth}{-2.3cm}{}
\begin{tabular}{l|c|c|c|c|c|c|c|c}
	& QT & RNJ & SN & QT/SN & RNJ/SN & RF(QT,SN) & RF(RNJ,SN) & RF(RNJ,QT) \\\hline
89\_Adeno\_E3\_CR1 & 12 & 10 & SN & QT/SN & RNJ/SN & RF(QT,SN) & RF(RNJ,SN) & RF(RNJ,QT) \\\hline
214\_Arena\_glycoprot & 29 & 23 & SN & QT/SN & RNJ/SN & RF(QT,SN) & RF(RNJ,SN) & RF(RNJ,QT) \\\hline
304\_A1\_Propeptide & 50 & 32 & SN & QT/SN & RNJ/SN & RF(QT,SN) & RF(RNJ,SN) & RF(RNJ,QT) \\\hline
401\_DDE & 84 & 47 & SN & QT/SN & RNJ/SN & RF(QT,SN) & RF(RNJ,SN) & RF(RNJ,QT) \\\hline
494\_Astro\_capsid & 123 & 60 & SN & QT/SN & RNJ/SN & RF(QT,SN) & RF(RNJ,SN) & RF(RNJ,QT) \\\hline
608\_Gemini\_AL2 & 205 & 108 & SN & QT/SN & RNJ/SN & RF(QT,SN) & RF(RNJ,SN) & RF(RNJ,QT) \\\hline
777\_Gemini\_V1 & 354 & 143 & SN & QT/SN & RNJ/SN & RF(QT,SN) & RF(RNJ,SN) & RF(RNJ,QT) \\\hline
877\_Glu\_synthase & 511 & 200 & SN & QT/SN & RNJ/SN & RF(QT,SN) & RF(RNJ,SN) & RF(RNJ,QT) \\\hline
1347\_FAINT & 2036 & 508 & SN & QT/SN & RNJ/SN & RF(QT,SN) & RF(RNJ,SN) & RF(RNJ,QT) \\\hline
1493\_Fe-ADH & 2384 & 599 & SN & QT/SN & RNJ/SN & RF(QT,SN) & RF(RNJ,SN) & RF(RNJ,QT) \\\hline
1560\_Ferritin & 2659 & 611 & SN & QT/SN & RNJ/SN & RF(QT,SN) & RF(RNJ,SN) & RF(RNJ,QT) \\\hline
1689\_FGGY\_N & 3184 & 747 & SN & QT/SN & RNJ/SN & RF(QT,SN) & RF(RNJ,SN) & RF(RNJ,QT) \\\hline
1756\_FAD\_binding\_3 & 3616 & 858 & SN & QT/SN & RNJ/SN & RF(QT,SN) & RF(RNJ,SN) & RF(RNJ,QT) \\\hline
1849\_FG-GAP & 4485 & 865 & SN & QT/SN & RNJ/SN & RF(QT,SN) & RF(RNJ,SN) & RF(RNJ,QT)
\end{tabular}
\caption{\label{tab:results}Results of all experiments}
\end{adjustwidth}
\end{figure}

\bibliography{references}

\end{document}


