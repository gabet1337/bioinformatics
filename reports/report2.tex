\documentclass[a4paper,oneside,article,11pt]{memoir}
\usepackage{changepage}
\usepackage[english]{babel}
\usepackage[utf8]{inputenc}
\usepackage{amsmath,amssymb,amsthm}

\newtheorem*{toprove}{At bevise}
\newtheorem{mydef}{Definition}

% This font looks so good.
\usepackage[sc]{mathpazo}
\usepackage{fancybox}
% Typesetting pseudo-code
\usepackage{algorithm}
\usepackage{algorithmic}
\usepackage{multirow}
% Code comments like [CLRS]
\renewcommand{\algorithmiccomment}[1]{\makebox[5cm][l]{$\triangleright$ \textit{#1}}}
\usepackage{framed,graphicx,xcolor}
\usepackage{listings}
\usepackage{hyperref}

\usepackage[font={small,it}]{caption}

% Relative references
\usepackage{varioref}

\usepackage{tikz}

\bibliographystyle{plain}

\title{Bioinformatics - Tree Comparison \\ Project 2}
\author{Peter Gabrielsen 20114179\\
Christoffer Hansen 20114637}
\newcounter{qcounter}

\makeatletter
\newenvironment{CenteredBox}{% 
\begin{Sbox}}{% Save the content in a box
\end{Sbox}\centerline{\parbox{\wd\@Sbox}{\TheSbox}}}% And output it centered
\makeatother

\begin{document}

\maketitle

\chapter*{Introduction}
%A short status of your work. Does everything work as expected, or are there any problems or unsolved issues.
In this project we are going to implement Saitou and Nei's algorithm for Neighbor joining. The aim of the project will be to make it run as fast as possible using different techniques and then compare our fastest implementation against \texttt{RapidNJ} and \texttt{QuickTree}.
Our implementation works as expected, but for one of our implementations we did not get as good a running time as expected.
\\\\Code can be found at \url{https://github.com/gabet1337/bioinformatics}.

\pagebreak

\chapter*{Implementation}
The implementations is done in \texttt{c++}. 
We implemented two different ideas; one naive with heave parallelization and one using the idea from the RapidNJ paper.

\subsection*{OpenMP version}
This version heavily relies on OpenMP to parallelize the naive version of Saitou-Nei's neighbor joining algorithm. The row sums and finding the minimum element in the N matrix is parallelized using OpenMP. When we remove columns and rows we just mark it as deleted and when we have deleted 10\% of the entire matrix we garbage collect the deleted rows and columns and globally rebuild the structure. We tried parallelizing the garbage collection but with a decrease in performance.

It compiles using \texttt{make nj} and runs using \texttt{./nj $<$phylib file$>$}.

As seen in~\ref{fig:openmp} our naive implementation with OpenMP parallization runs within a factor of two of QuickTree. These results were found by repeatedly running the programs and average the running time over 10 runs.

\begin{figure}[H]
\includegraphics[scale=0.5]{../results/running_time_all.png}
\caption{\label{fig:openmp}Running time of OpenMP version compared to QuickTree and RapidNJ}
\end{figure}

\subsection*{Machine}
Running on an i7-3720QM @ 2.60GHz with 4GB ram in a VMWare Ubuntu 14.04 with kernel 3.16.0-50-generic. Compiled with optimization level 3.

\subsection*{Results }
\begin{figure}[H]
\begin{adjustwidth}{-2.3cm}{}
\begin{tabular}{l|c|c|c|c|c|c|c|c}
						& QT 	& RNJ 	& SN		& QT/SN 		& RNJ/SN & RF(QT,SN)	& RF(RNJ,SN) & RF(RNJ,QT) 	\\\hline
89\_Adeno\_E3\_CR1 		& 5 		& 4 		& 11   	& 0.45	 	& 0.36	 & 18		& 34			 & 30 			\\\hline
214\_Arena\_glycoprot 	& 19 	& 12 	& 69   	& 0.27 		& 0.27	 & 36		& 50			 & 56 			\\\hline
304\_A1\_Propeptide 		& 38 	& 20 	& 98   	& 0.39 		& 0.20	 & 38		& 74			 & 78 			\\\hline
401\_DDE 				& 66 	& 35 	& 150  	& 0.44 		& 0.23	 & 20		& 98			 & 98 			\\\hline
494\_Astro\_capsid 		& 96 	& 44 	& 244  	& 0.39 		& 0.18	 & 244		& 530		 & 532 			\\\hline
608\_Gemini\_AL2 		& 161 	& 91 	& 391  	& 0.41	 	& 0.23	 & 16		& 20			 & 20 			\\\hline
777\_Gemini\_V1 			& 288 	& 123 	& 668  	& 0.43 		& 0.18	 & 148		& 480		 & 494 			\\\hline
877\_Glu\_synthase 		& 420 	& 178 	& 893  	& 0.47 		& 0.20	 & 34		& 60			 & 54 			\\\hline
1347\_FAINT 				& 1660 	& 445	& 2570 	& 0.65 		& 0.17	 & 0			& 2			 & 2 			\\\hline
1493\_Fe-ADH 			& 1919 	& 508 	& 3349 	& 0.57 		& 0.15	 & 10		& 88			 & 92 			\\\hline
1560\_Ferritin 			& 2177 	& 534 	& 3803 	& 0.57 		& 0.14	 & 40		& 146		 & 152 			\\\hline
1689\_FGGY\_N 			& 2636 	& 652 	& 4623 	& 0.57 		& 0.14	 & 26		& 104		 & 102 			\\\hline
1756\_FAD\_binding\_3 	& 2981 	& 720 	& 4890 	& 0.61 		& 0.15	 & 40		& 68			 & 76 			\\\hline
1849\_FG-GAP 			& 3536 	& 729 	& 5675 	& 0.62 		& 0.13	 & 132		& 282		 & 280
\end{tabular}
\caption{\label{tab:results}Results of all experiments}
\end{adjustwidth}
\end{figure}

\subsection*{RapidNJ idea}
We decided to implement Saitou-Nei's algorithm (SN) using the RapidNJ search heuristic to speed-up the computations of the neighbour-joining method. Even though we did a naive implementation of the algorithm, using no clever way of computing row sums and reuse the distance matrix, we would have expected that the heuristic would have provided a significant speed-up compared against the standard SN algorithm. This was not what we saw when testing, as we achieved a running time of more than 19 minutes on 1849 taxons. We suspected the implementation to be buggy with respect to the pruning of the computations, but running the implementation on small examples and tracking the actual pruning being done, gave no such evidence. Furthermore we were able to test correctness of our implementation against both QT and RNJ against large test-input (using RF-distance as measure), so the best we can conclude is that the implementation must have been buggy in other parts than the pruning step.

\chapter*{Conclusion}
We did not manage to achieve a competing implementation using the \textit{clean} RapidNJ idea. We were able to come within a factor of $0.6$ of QuickTree while RapidNJ becomes more and more fast compared to our implementation utilizing OpenMP.

\bibliography{references}

\end{document}


